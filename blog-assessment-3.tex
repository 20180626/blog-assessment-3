%% LyX 2.3.6 created this file.  For more info, see http://www.lyx.org/.
%% Do not edit unless you really know what you are doing.
\documentclass[a4paper,british]{article}
\usepackage[utf8]{inputenc}
\pagestyle{plain}
\setlength{\parindent}{4sp}
\usepackage[active]{srcltx}
\usepackage{amsmath}
\usepackage{setspace}
\onehalfspacing

\makeatletter

%%%%%%%%%%%%%%%%%%%%%%%%%%%%%% LyX specific LaTeX commands.
\special{papersize=\the\paperwidth,\the\paperheight}

\DeclareTextSymbolDefault{\textquotedbl}{T1}

\makeatother

\usepackage{babel}
\begin{document}
\title{\textbf{\textquotedbl Trust but Verify\textquotedbl : Addressing The
Risks of Cyber-Physical Systems}}
\author{Group 4}
\date{\noindent \emph{Date}}
\maketitle
\begin{onehalfspace}

\section{Looking at BS 10754-1:2018, how does the proposed governance approach
classify and manage risk to ensure systems trustworthiness? Discuss
this with reference to cyber-physical systems as addressed in Session
5}
\end{onehalfspace}
\begin{enumerate}
\item Governance
\item Risk
\item Controls
\item Compliance
\end{enumerate}
Information asymmetry between CPS operators and general public

Principle-agent problem - public-private partnerships

Mismorphism

\section{Review the risk assessment techniques proposed in IEC 31010:2019,
Sections 6.3-6.6 and Annex B. Select two techniques from Annex B,
summarise them and discuss their relevance, benefits and limitations. }

\subsection*{Bayesian Analysis}

It is common to encounter problems where there is both data and subjective
information. Bayesian analysis enables both types of information to
be used in making decisions. Bayesian analysis is based on a theorem
attributed to Reverend Thomas Bayes (1760). At its simplest, Bayes's
theorem provides a probabilistic basis for changing one's opinion
in the light of new evidence.
\begin{figure}
\textbf{\caption{Bayes Theorem}
}

\texttt{
\[
P(A|B)=\dfrac{P(B|A)P(A)}{P(B)}
\]
}
\end{figure}

\textbackslash{}

\subsection*{Game Theory}

Game theory is a means to model the consequences of different possible
decisions given a number of possible future situations. The future
situations can be determined by a different decision maker (e.g. a
competitor) or by an external event, such as success or failure of
a technology or a test. For example, assume the task is to determine
the price of a product taking into account the different decisions
that could be made by different decision makers (called players) at
different times. The payoff for each player involved in the game,
relevant to the time period concerned, can be calculated and the strategy
with the optimum payoff for each player selected. Game theory can
also be used to determine the value of information about the other
player or the different possible outcomes (e.g. success of a technology).
There are different types of games, for example cooperative/non--cooperative,
symmetric/asymmetric, zero-sum/non-zero-sum, simultaneous/sequential,
perfect information and imperfect information, combinatorial games,
stochastic outcomes.

TRUST BUT VERIFY
\begin{thebibliography}{1}
\bibitem{key-1}
\end{thebibliography}

\end{document}
